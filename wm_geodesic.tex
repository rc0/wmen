\documentclass[10pt,a4paper,twocolumn]{article}
\usepackage{charter}
%\usepackage{inconsolata}
\usepackage{geometry}
\usepackage{graphicx}
\usepackage{caption}
%\usepackage{subcaption}
\usepackage{rotating}
\usepackage{gensymb}
\usepackage{lettrine}
\usepackage{tikz}
\usetikzlibrary{arrows,positioning,shapes}

%%\usetikzlibrary{external}
%%\tikzexternalize[
%%  mode=list and make,
%%  prefix=figures/]
%%\def\tkname#1{\tikzsetnextfilename{#1}}
\def\tkname#1{}

\usepackage{booktabs}
\geometry{left=0.5in,right=0.5in,top=0.5in,bottom=1in,a4paper}
\setlength{\parskip}{2ex plus 0.5ex minus 0.5ex}
\setlength{\parindent}{0pt}
\def\gap{\hspace{2cm}}
\usepackage[charter]{mathdesign}
\usepackage{amsmath}
\usepackage[draft]{hyperref}
\title{Fast approximation of geodesic distance between\\points specified by Web Mercator coordinates}
\author{Richard P. Curnow}

\tikzset{
    %Define standard arrow tip
    >=stealth',
    %Define style for boxes
    box/.style={
           rectangle,
           rounded corners,
           draw=black, very thick,
           text width=9em,
           minimum height=2em,
           text centered},
    boxzag/.style={
           draw=black, very thick,
           starburst,
           text width=6.5em,
           minimum height=2em,
           text centered},
    annot/.style={
           text width=7em,
           text centered},
    % Define arrow style
    arr/.style={
           ->,
           very thick,
           shorten <=2pt,
           shorten >=2pt,},
    redarr/.style={
           ->,
           very thick,
           draw=red,
           shorten <=2pt,
           shorten >=2pt,},
    bluearr/.style={
           ->,
           very thick,
           draw=blue,
           shorten <=2pt,
           shorten >=2pt,},
    purparr/.style={
           ->,
           very thick,
           draw=blue!50!red,
           shorten <=2pt,
           shorten >=2pt,}
}


\begin{document}
\maketitle

\section {Background}
\lettrine{W}{hen} writing mapping applications \cite{app} that use map tiles
from the Internet (e.g.  OpenStreetMap), it is convenient to work in the
\textit{Web Mercator} coordinates.  These can be related easily to the tiles
coordinate system.  If a need arises to compute an approximate geodesic
distance between two points defined by their Web Mercator coordinates, the
formulae given in this note may be useful.

This note is placed under a Creative Commons CC-BY-SA licence.  The algorithm
described in section \ref{secn:formula} is placed under a CC-BY licence.

\section {Terminology}
\textit{Web Mercator} is the system used for indexing the tiles on web-based
map servers such as OpenStreetMap.  A location's coordinates are expressed as
$(X,Y)$ where both $X$ and $Y$ lie in $[0,1]$.  In software making extensive
use of map tiles from the web, it is particularly efficient to represent
locations in this form.  If map tiles at zoom level $z$ are being accessed, the
value $(2^zX)$ may be written as $t_x . f_x$, where $t_x$ is the integer
part and $0.f_x$ is the fractional part.  $t_y$ and $f_y$ can be defined
similarly.  Then the map tile containing the location is $(t_x,t_y)$ and the
location is $0.f_x$ in from the left side of the tile and $0.f_y$ down from the
top.

To calculate the Web Mercator coordinates $(X,Y)$ from latitude $M$ and
longitude $L$, in \textbf{radians} with north and east positive, evaluate

\begin{align}
  X & = \frac{1}{2}\left(1 + \frac{L}{\pi} \right) \label{eqn:wmx} \\
  Y & = \frac{1}{2} \left(
        1 - \frac{1}{\pi} \ln \left[
          \tan\left(\frac{M}{2} + \frac{\pi}{4}\right)
        \right]
        \right)
\end{align}

These formulae can be inverted by

\begin{align}
  L &= (2X - 1)\pi \\
  M &= \left[2\tan^{-1}\left(e^t\right)\right] - \frac{\pi}{2}
  \label{eqn:y_to_M}
  \intertext{where}
  t &= (1 - 2Y)\pi
\end{align}

\section {Approximate distance formula}
\label{secn:formula}

Given two points defined in Web Mercator by $(X_0,Y_0)$ and $(X_1,Y_1)$, the
distance $D$ between them can be approximated using the method below:

\begin{align}
  \delta_X & = X_1 - X_0 \\
  \delta_Y & = Y_1 - Y_0 \\
  v & = \frac{Y_0 + Y_1}{2} - \frac{1}{2} \\
  \delta_Y ' & = \left ( J_0 + J_2 v^2 + J_4 v^4 \right) \delta_Y
\end{align}

then

\begin{equation}
  \boxed{
    D \approx \frac{2 \pi a \sqrt{\delta_X^2 + \delta_Y'^2}}{1 + K_2 v^2 + K_4 v^4}
  }
\end{equation}


where the symbols have the following meanings:

\begin{table}[h]
\centering
\begin{tabular}{c p{2.5in}}
\toprule
\textbf{Symbol} & \textbf{Meaning} \\
\midrule
$v$ & translated Y-coordinate of the midpoint (Equator at zero) \\
$\delta_Y'$ & adjusted difference between the Y-coordinates \\
$a$ & Equatorial radius of the Earth, 6378137.0 metres \\
$J_i$, $K_i$ & suitable constants as follows \\
$J_0$ & 0.99330562 \\
$J_2$ & 0.18663111 \\
$J_4$ & -1.45510549 \\
$K_2$ & 19.42975297 \\
$K_4$ & 74.22319781 \\
$D$ & Estimated distance between $(X_0,Y_0)$ and $(X_1,Y_1)$, in metres \\
\bottomrule
\end{tabular}
\end{table}

The approximation is accurate to 0.1\% up to 71 degrees latitude and for
distances up to at least (200 $\cos(M)$) km.  As the formulae are intended for
machine implementation, the $J_i$, $K_i$ values are as obtained in the method
described below, without rounding.

\section {Derivation}
For a spherical Earth, the scaling to apply to $\delta_X$ would be the distance
round the parallel of latitude $M$, which is $2\pi a \cos M$.  Since the Earth
is ellipsoidal rather than spherical, the flattening at the poles needs to be
included too.  If $e$ is the eccentricity of the Earth, the scaling $S_X$ to
apply to $\delta_X$ is given by

\begin{align}
S_X & = \frac {2\pi a \cos M}{\sqrt{1 - e^2 \sin^2M}}
\end{align}

The latitude $M$ is not directly available, because the calculation is starting
from $(X,Y)$ Web Mercator coordinates.  So the objective is to approximate
$S_X(Y)$ given $Y$, defined by the equations

\begin{align}
t(Y) &= (1 - 2Y)\pi \\
M(Y) &= \left[2\tan^{-1}\left(e^t(Y)\right)\right] - \frac{\pi}{2} \\
S_X(Y) & = \frac {2\pi a \cos M(Y)}{\sqrt{1 - e^2 \left(\sin M(Y)\right)^2}}
\end{align}

Initially the approximation was attempted for $Y$ anywhere in $[0,1]$.  The
approximation was poor near to the polar regions.  Attempting to reduce this
inaccuracy impacted the accuracy at lower latitude.  Therefore the
approximation was restricted to $Y$ in the range $[0.215,0.785]$.  This covers
the range of latitudes up to about 71 degrees north and south.

A degree-n polynomial was fitted to this function.  The procedure was to
calculate the Chebyshev nodes for degree-n in the range $[-1,+1]$, map these
into the interval $[0.215,0.785]$ via the mapping $r_n \mapsto (0.5 +
0.285r_n)$, then find the coefficients by solving a set of n+1 linear
equations. The target is to match the scaling $S_X(Y)$ exactly at the chosen
points.

The Chebyshev nodes are given by
\begin{align}
  r_i & = \cos \left(
  \frac{\left(2i-1\right)\pi}{2n}
  \right)
\end{align}

for $i = 1 \cdots n$.  So for $n=5$, the nodes and the corresponding Web
Mercator $Y_i$ values are given by

\begin{table}[h]
  \centering
  \begin{tabular}{c c c}
    \toprule
    $i$ & $r_i$ & $Y_i$ \\
    \midrule
    1 & 0.95106 & 0.77105 \\
    2 & 0.58779 & 0.66752\\
    3 & 0.00000 & 0.50000 \\
    4 & -0.58779 & 0.33248 \\
    5 & -0.95106  & 0.22895 \\
    \bottomrule
  \end{tabular}
\end{table}

Unfortunately as n was increased, the coefficients became larger and larger,
indicating that $S_X$ was a difficult function to approximate with a
polynomial.  However, by re-casting $S_X$ as

\begin{align}
S_X'(Y) & = \frac {\sqrt{1 - e^2 \left(\sin M(Y)\right)^2}}{\cos M(Y)} \\
S_X(Y) & = \frac {2\pi a}{S_X'(Y)}
\end{align}

it was observed that $1/S_X'(Y)$ has a bell shape, which suggested that an
approximation of a form like

\begin{align}
\frac{1}{S_X'(Y)} & \approx \frac{1}{1 + K\Psi^2}
\end{align}

might be useful for some appropriate $\Psi$.  This means that $S_X'(Y)$ itself
can be approximated by a polynomial.  The expression is symmetrical in
latitude, so we can substitute $\Psi = (Y - 0.5)$,  given that $Y = 0.5$ at the
Equator.  By trial and error, the value $K$ is found to be around 19.  After
solving the set of 5 linear equations to give an exact fit at the Chebyshev
nodes, a polynomial approximation in $\Psi$ with only even powers is reached,
as follows:

\begin{align}
S_X'(Y) & \approx 1 + K_2 \Psi^2 + K_4 \Psi^4
\end{align}

This is the source of the $K_2$ and $K_4$ constants earlier.  Obviously the
$K_1$ and $K_3$ terms solve very close to zero --- a useful cross-check on the
implementation.

If the same scaling is applied to a pure $\delta_Y$ displacement, the match
against the exact distance given by Vincenty's formulae \cite{vincenty} is much
worse than for pure $\delta_X$ terms.  An additional scaling to apply to the
$\delta_Y$ term was sought to see if this would correct the error.  The scaling
required is also symmetrical about the Equator.  It turns out that the three
even-order terms of a different quartic in $\Psi$ with suitable coefficients
give a satisfactory approximation.  Fitting at the same $Y_i$ values as before
yields the $J_0$, $J_2$ and $J_4$ values given earlier to compute $\delta_Y'$
from $\delta_Y$.  Pythagoras's theorem is then used to aggregate the $\delta_X$
and $\delta_Y'$ contributions to deal with arbitary azimuths of the
displacement.  This was found to introduce negligible further errors.

The source code used to generate and analyse this method can be found at
\cite{source}.

\begin{thebibliography}{99}
  \bibitem {vincenty} \url{http://en.wikipedia.org/wiki/Vincenty%27s_formulae}
    \bibitem {source} \url{https://github.com/rc0/wmen}
  \bibitem {app} \url{https://github.com/rc0/logmygsm}
\end{thebibliography}

\end{document}

% vim:et:sw=2:sts=2:ts=2:ht=2

